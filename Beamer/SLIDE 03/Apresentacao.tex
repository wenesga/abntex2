\documentclass[svgnames,12pt,aspectratio=169]{beamer}
\mode<presentation>
{
  \usetheme{Boadilla}
  \setbeamercovered{dynamic}
}

\usepackage[brazil]{babel}
\usepackage[utf8]{inputenc}
\usepackage[T1]{fontenc}
\usepackage{Estilo/BeamerLille}

% para melhorar copiar/colar
\usepackage{cmap}
\usepackage{lmodern}

\usepackage{graphicx}
\graphicspath{ {./Imagens/} }
\usepackage{tikz}
\usepackage{amssymb,amsfonts}

\author{Wenes Gomes Aquino}
\title{Título da apresentação}
\subtitle{Subtítulo da apresentação}

\begin{document}

\begin{frame}[plain]
  \titlepage
\end{frame}

\begin{frame}
\section*{Sumário}
\frametitle{Sumário}
\tableofcontents
\end{frame}


\begin{frame}
\section{Explicação}
    \frametitle{Explicação}
    \begin{columns}
      \begin{column}{0.5\textwidth}
          Primeira coluna para preencher
    
    \begin{enumerate}
        \item Item 1
        \item Item 2
        \item Item 3
    \end{enumerate}
          \end{column}
          
          \begin{column}{0.5\textwidth}
              Segunda coluna para preencher
    
    \begin{itemize}
        \item Item 1
        \item Item 2
        \item Item 3
    \end{itemize}
          \end{column}
        \end{columns}
\end{frame}


\begin{frame}
    \section{Ambientes}
    \frametitle{Ambientes}
    \begin{block}{Teste: Bloco}
        Este é um bloco
    \end{block}
    
    \begin{alertblock}{Teste: Bloco de Alerta}
        Este é um bloco
    \end{alertblock}
    
    \begin{exampleblock}{Teste: Bloco de Exemplo}
        Este é um bloco
    \end{exampleblock}
    
    Como pode ver, os \alert{blocos} e \alert{blocos de alerta} são bem parecidos.
\end{frame}

\end{document} 