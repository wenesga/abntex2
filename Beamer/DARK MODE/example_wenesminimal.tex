\documentclass[aspectratio=169]{beamer}

\usepackage[dark]{beamerthemeWenesMinimal}
\usepackage{graphicx}
\usepackage{hyperref}
\usepackage[brazil]{babel}

\title{\textsc{Apresentação Dark}}
\author{\textsc{Wenes Gomes Aquino}}
\date{2025}

\begin{document}

% TÍTULO
\begin{frame}
  \titlepage
\end{frame}

% SLIDE 1 – Texto simples
\begin{frame}{\textsc{Modo Dark Ativado}}
\Large
O tema agora está totalmente otimizado para ambientes escuros.

\bigskip
✔ Contraste perfeito  
✔ Small caps funcionando  
✔ Estética moderna  
✔ Zero erros  
\end{frame}

% SLIDE 2 – Dois blocos lado a lado
\begin{frame}{\textsc{Dois Blocos}}
\begin{columns}[T]
  \column{0.48\textwidth}
    \begin{block}{\textsc{Exemplo 1}}
      Conteúdo de teste no estilo escuro.
    \end{block}

  \column{0.48\textwidth}
    \begin{block}{\textsc{Exemplo 2}}
      Tudo limpo, elegante e balanceado.
    \end{block}
\end{columns}
\end{frame}

% SLIDE 3 – Dois slides na mesma tela
\begin{frame}{\textsc{Dois “Slides” em um}}
\begin{columns}[T]
  \column{0.48\textwidth}
    \large\textbf{Mini-Slide A}\par
    Legibilidade aprimorada.

  \column{0.48\textwidth}
    \large\textbf{Mini-Slide B}\par
    Layout ideal pra resumo visual.
\end{columns}
\end{frame}

% SLIDE 4 – Bloco de destaque
\begin{frame}
\begin{block}{\textsc{Destaque}}
Dark Mode deixa tudo mais elegante, tecnológico e limpo.
\end{block}
\end{frame}

% SLIDE 5 – Imagem
\begin{frame}[plain]
  \centering
  \includegraphics[width=\paperwidth]{example-image}
\end{frame}

\end{document}
