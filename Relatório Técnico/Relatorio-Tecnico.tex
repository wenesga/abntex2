% -----------------------------------------------
% Modelo UFT para Trabalhos Acadêmicos
% Adaptado e mantido por Wenes (Libertário)
% Baseado no projeto abnTeX2, porém modificado para atender às diretrizes
% visuais e estruturais da Universidade Federal do Tocantins (UFT).
% Esta versão não é oficial e não substitui os modelos originais do abnTeX2.
% -----------------------------------------------

\documentclass[
	12pt,				% Tamanho da fonte
    oneside,            % Garante margens iguais (não alternadas)
	a4paper,			% tamanho do papel. 
	english,			% idioma adicional para hifenização
	french,				% idioma adicional para hifenização
	spanish,			% idioma adicional para hifenização
	brazilian,		    % o último idioma é o principal do documento
	]{abntex2}

% -----------------------------------------------
% PACOTES
% -----------------------------------------------

% -----------------------------------------------
% Pacotes fundamentais 
% -----------------------------------------------
\usepackage{lmodern}			% Usa a fonte Latin Modern
\usepackage[T1]{fontenc}		% Selecao de codigos de fonte.
\usepackage[utf8]{inputenc}		% Codificacao do documento (conversão automática dos acentos)
\usepackage{indentfirst}		% Indenta o primeiro parágrafo de cada seção.
\usepackage{color}				% Controle das cores
\usepackage{graphicx}			% Inclusão de gráficos
\usepackage{microtype} 			% para melhorias de justificação
% -----------------------------------------------

% -----------------------------------------------
% Pacotes adicionais, usados no anexo do modelo de folha de identificação
% -----------------------------------------------
\usepackage{multicol}
\usepackage{multirow}
% -----------------------------------------------
	
% -----------------------------------------------
% Pacotes adicionais, usados apenas no âmbito do Modelo Canônico do abnteX2
% -----------------------------------------------
\usepackage{lipsum}	% para geração de dummy text
% -----------------------------------------------

% -----------------------------------------------
% Pacotes de citações
% -----------------------------------------------
\usepackage[brazilian,hyperpageref]{backref}	 % Paginas com as citações na bibl
\usepackage[alf]{abntex2cite}	% Citações padrão ABNT

% -----------------------------------------------
% CONFIGURAÇÕES DE PACOTES
% -----------------------------------------------







% -----------------------------------------------
% Informações de dados para CAPA e FOLHA DE ROSTO
% -----------------------------------------------
\titulo{Modelo UFT para\\ Relatório Técnico em \abnTeX}
\autor{Editado por Wenes Aquino}
\local{Tocantins}
\data{2025, v-1.0}
\instituicao{%
  Universidade Federal do Tocantins - UFT\\
  Campus Universitário de Palmas\\
  Curso de Licenciatura em Computação}
\tipotrabalho{Relatório Técnico}
% O preambulo deve conter o tipo do trabalho, o objetivo, 
% o nome da instituição e a área de concentração 
\preambulo{Trabalho acadêmico apresentado como parte dos requisitos para avaliação na Universidade Federal do Tocantins (UFT). Este modelo foi adaptado e personalizado com ajustes visuais, estruturais e técnicos, servindo também para testar o espaço disponível e a formatação utilizada na contracapa deste documento.}
% -----------------------------------------------



\makeatletter
\hypersetup{
    colorlinks=false,   % Desativa links coloridos
    pdfborder={0 0 0},  % Remove o retângulo ao redor dos links
}
\makeatother
% -----------------------------------------------

% -----------------------------------------------
% Espaçamentos entre linhas e parágrafos 
% -----------------------------------------------

% O tamanho do parágrafo é dado por:
\setlength{\parindent}{1.3cm}

% Controle do espaçamento entre um parágrafo e outro:
\setlength{\parskip}{0.2cm}  % tente também \onelineskip

% -----------------------------------------------
% compila o indice
% -----------------------------------------------
\makeindex
% -----------------------------------------------

% -----------------------------------------------
% Início do documento
% -----------------------------------------------
\begin{document}

\selectlanguage{brazil}

% Retira espaço extra obsoleto entre as frases.
\frenchspacing 

% -----------------------------------------------
% ELEMENTOS PRÉ-TEXTUAIS
% -----------------------------------------------
% \pretextual

% -----------------------------------------------
% Capa
% -----------------------------------------------
\imprimircapa
% -----------------------------------------------

% -----------------------------------------------
% Folha de rosto
% (o * indica que haverá a ficha bibliográfica)
% -----------------------------------------------
\imprimirfolhaderosto*
% -----------------------------------------------

% -----------------------------------------------
% Anverso da folha de rosto:
% -----------------------------------------------


% -----------------------------------------------
% Agradecimentos
% -----------------------------------------------
\begin{agradecimentos}
\lipsum[1-2] % Gera texto de exemplo para preencher a página.
\footnote{\url{http://www.cpai.unb.br/}} 

\end{agradecimentos}
% -----------------------------------------------



% -----------------------------------------------
% RESUMO
% -----------------------------------------------
% resumo na língua vernácula (obrigatório)
\setlength{\absparsep}{18pt} % ajusta o espaçamento dos parágrafos do resumo
\begin{resumo}
\lipsum[1-2] % Gera texto de exemplo para preencher a página.

 \noindent
 \textbf{Palavras-chaves}: latex. abntex. editoração de texto.
\end{resumo}
% -----------------------------------------------

% -----------------------------------------------
% inserir lista de ilustrações
% -----------------------------------------------
\pdfbookmark[0]{\listfigurename}{lof}
\listoffigures*
\cleardoublepage
% -----------------------------------------------

% -----------------------------------------------
% inserir lista de tabelas
% -----------------------------------------------
\pdfbookmark[0]{\listtablename}{lot}
\listoftables*
\cleardoublepage
% -----------------------------------------------

% -----------------------------------------------
% inserir lista de abreviaturas e siglas
% -----------------------------------------------
\begin{siglas}
  \item[ABNT] Associação Brasileira de Normas Técnicas
  \item[abnTeX] ABsurdas Normas para TeX
\end{siglas}
% -----------------------------------------------

% -----------------------------------------------
% inserir lista de símbolos
% -----------------------------------------------
\begin{simbolos}
  \item[$ \Gamma $] Letra grega Gama
  \item[$ \Lambda $] Lambda
  \item[$ \zeta $] Letra grega minúscula zeta
  \item[$ \in $] Pertence
\end{simbolos}
% -----------------------------------------------

% -----------------------------------------------
% inserir o sumario
% -----------------------------------------------
\pdfbookmark[0]{\contentsname}{toc}
\tableofcontents*
\cleardoublepage
% -----------------------------------------------


% -----------------------------------------------
% ELEMENTOS TEXTUAIS
% -----------------------------------------------
\textual

% -----------------------------------------------
% Introdução (exemplo de capítulo sem numeração, mas presente no Sumário)
% -----------------------------------------------
\chapter*[Introdução]{Introdução}
\addcontentsline{toc}{chapter}{Introdução}

\lipsum[1-2] % Gera texto de exemplo para preencher a página.







% -----------------------------------------------
% PARTE - preparação da pesquisa
% -----------------------------------------------
\part{Preparação do relatório}



% -----------------------------------------------
% Parte de resultados
% -----------------------------------------------
\part{Resultados}

% -----------------------------------------------
% Capitulo de revisão de literatura
% -----------------------------------------------
\chapter{Lorem ipsum dolor sit amet}

% -----------------------------------------------
\section{Aliquam vestibulum fringilla lorem}
% -----------------------------------------------
\lipsum[1-2] % Gera texto de exemplo para preencher a página.





% -----------------------------------------------
% Finaliza a parte no bookmark do PDF
% para que se inicie o bookmark na raiz
% e adiciona espaço de parte no Sumário
% -----------------------------------------------
\phantompart

% -----------------------------------------------
% Conclusão
% -----------------------------------------------
\chapter{Conclusão}
% -----------------------------------------------
\lipsum[1-2] % Gera texto de exemplo para preencher a página.


% -----------------------------------------------
% ELEMENTOS PÓS-TEXTUAIS
% -----------------------------------------------
\postextual

% -----------------------------------------------
% Referências bibliográficas
% -----------------------------------------------
\bibliography{Referencias}

% -----------------------------------------------
% Glossário
% -----------------------------------------------
%
% Consulte o manual da classe abntex2 para orientações sobre o glossário.
%
%\glossary

% -----------------------------------------------
% Apêndices
% -----------------------------------------------

% ---
% Inicia os apêndices
% ---
\begin{apendicesenv}

% Imprime uma página indicando o início dos apêndices
\partapendices

% -----------------------------------------------
\chapter{Quisque libero justo}
\lipsum[1-2] % Gera texto de exemplo para preencher a página.


% -----------------------------------------------
\chapter{Nullam elementum urna vel imperdiet sodales elit ipsum}
\lipsum[1-2] % Gera texto de exemplo para preencher a página.


\end{apendicesenv}
% -----------------------------------------------


% -----------------------------------------------
% Anexos
% -----------------------------------------------

% -----------------------------------------------
% Inicia os anexos
% -----------------------------------------------
\begin{anexosenv}

% Imprime uma página indicando o início dos anexos
\partanexos

% ---
\chapter{Morbi ultrices rutrum lorem.}
\lipsum[1-2] % Gera texto de exemplo para preencher a página.

% ---
\chapter{Cras non urna sed feugiat cum sociis natoque penatibus}
\lipsum[1-2] % Gera texto de exemplo para preencher a página.

% ---
\chapter{Fusce facilisis lacinia dui}
\lipsum[1-2] % Gera texto de exemplo para preencher a página.

\end{anexosenv}

% -----------------------------------------------
% INDICE REMISSIVO
% -----------------------------------------------

\phantompart

\printindex

% -----------------------------------------------
% Formulário de Identificação (opcional)
% -----------------------------------------------
\chapter*[Formulário de Identificação]{Formulário de Identificação}
\addcontentsline{toc}{chapter}{Exemplo de Formulário de Identificação}
\label{formulado-identificacao}

Exemplo de Formulário de Identificação, compatível com o Anexo A (informativo)
da ABNT NBR 10719:2015. Este formulário não é um anexo. Conforme definido na
norma, ele é o último elemento pós-textual e opcional do relatório.

\bigskip

\begin{tabular}{|p{9cm}|p{5cm}|}
\hline
\multicolumn{2}{|c|}{\textbf{\large Dados do Relatório Técnico e/ou científico}}\\
\hline
\multirow{4}{10cm}[24pt]{Título e subtítulo}& Classificação de segurança\\
                   & \\
                   \cline{2-2}
                   & No.\\
                   & \\
				
\hline
Tipo de relatório & Data\\
\hline
Título do projeto/programa/plano & No.\\
\hline
\multicolumn{2}{|l|}{Autor(es)} \\
\hline
\multicolumn{2}{|l|}{Instituição executora e endereço completo} \\
\hline
\multicolumn{2}{|l|}{Instituição patrocinadora e endereço completo} \\
\hline
\multicolumn{2}{|l|}{Resumo}\\[3cm]
\hline
\multicolumn{2}{|l|}{Palavras-chave/descritores}\\
\hline
\multicolumn{2}{|l|}{
Edição \hfill No. de páginas \hfill No. do volume \hfill Nº de classificação \phantom{XXXX}} \\
\hline
\multicolumn{2}{|l|}{
ISSN \hfill \hfill Tiragem \hfill Preço \phantom{XXXXXXXX}} \\
\hline
\multicolumn{2}{|l|}{Distribuidor} \\
\hline
\multicolumn{2}{|l|}{Observações/notas}\\[3cm]
\hline
\end{tabular}

\end{document}
