% -----------------------------------------------
% Modelo UFT para Trabalhos Acadêmicos
% Adaptado e mantido por Wenes (Libertário)
% Baseado no projeto abnTeX2, porém modificado para atender às diretrizes
% visuais e estruturais da Universidade Federal do Tocantins (UFT).
% Esta versão não é oficial e não substitui os modelos originais do abnTeX2.
% -----------------------------------------------

\documentclass[
	12pt,				% Tamanho da fonte
    oneside,            % Garante margens iguais (não alternadas)
	a4paper,			% tamanho do papel. 
	english,			% idioma adicional para hifenização
	french,				% idioma adicional para hifenização
	spanish,			% idioma adicional para hifenização
	brazilian,		    % o último idioma é o principal do documento
	]{abntex2}

% -----------------------------------------------
% PACOTES
% -----------------------------------------------

% -----------------------------------------------
% Pacotes fundamentais 
% -----------------------------------------------
\usepackage{lmodern}			% Usa a fonte Latin Modern
\usepackage[T1]{fontenc}		% Selecao de codigos de fonte.
\usepackage[utf8]{inputenc}		% Codificacao do documento (conversão automática dos acentos)
\usepackage{indentfirst}		% Indenta o primeiro parágrafo de cada seção.
\usepackage{color}				% Controle das cores
\usepackage{graphicx}			% Inclusão de gráficos
\usepackage{microtype} 			% para melhorias de justificação
% -----------------------------------------------

% -----------------------------------------------
% Pacotes adicionais, usados apenas no âmbito do Modelo Canônico do abnteX2
% -----------------------------------------------
\usepackage{lipsum}	% para geração de dummy text
% -----------------------------------------------

% -----------------------------------------------
% Pacotes de citações
% -----------------------------------------------
\usepackage[backend=biber,style=abnt,language=brazil]{biblatex}
\addbibresource{Referencias.bib}
\usepackage{csquotes}             % Recomendado pelo biblatex 
\usepackage{etoolbox}             % Auxilia na justificação da bibliografia
\AtBeginBibliography{\justifying} % Ativa a justificação da bibliografia
\usepackage{ragged2e}


% -----------------------------------------------
% Informações de dados para CAPA e FOLHA DE ROSTO
% -----------------------------------------------
\titulo{Modelo UFT para Projeto de Pesquisa em \abnTeX}
\autor{Editado por Wenes Aquino}
\local{Tocantins}
\data{2015, v-1.0}
\instituicao{%
  Universidade Federal do Tocantins - UFT\\
  Campus Universitário de Palmas\\
  Curso de Licenciatura em Computação}
\tipotrabalho{Tese (Doutorado)}
% O preambulo deve conter o tipo do trabalho, o objetivo, 
% o nome da instituição e a área de concentração 
\preambulo{Trabalho acadêmico apresentado como parte dos requisitos para avaliação na Universidade Federal do Tocantins (UFT).  
Este modelo foi adaptado e personalizado com ajustes visuais, estruturais e técnicos, servindo também para testar o espaço disponível e a formatação utilizada na contracapa deste documento.
}
% -----------------------------------------------

% -----------------------------------------------
% Configurações de aparência do PDF final

% alterando o aspecto da cor azul
%\definecolor{blue}{RGB}{41,5,195}

% informações do PDF
\makeatletter
\hypersetup{
    colorlinks=false,   % desativa cores
    linkcolor=black,
    citecolor=black,
    filecolor=black,
    urlcolor=black
}
\makeatother

% -----------------------------------------------

% -----------------------------------------------
% Espaçamentos entre linhas e parágrafos 
% -----------------------------------------------

% O tamanho do parágrafo é dado por:
\setlength{\parindent}{1.3cm}

% Controle do espaçamento entre um parágrafo e outro:
\setlength{\parskip}{0.2cm}  % tente também \onelineskip

% -----------------------------------------------
% compila o indice
% -----------------------------------------------
\makeindex
% -----------------------------------------------




% -----------------------------------------------
\begin{document} % Início do documento
% -----------------------------------------------
\nocite{*} % Inclui todas referências na página
\selectlanguage{brazil}

% Retira espaço extra obsoleto entre as frases.
\frenchspacing 



% -----------------------------------------------
% ELEMENTOS PRÉ-TEXTUAIS
% -----------------------------------------------
% \pretextual


% -----------------------------------------------
% ----------FOLHA DE ROSTO E CONTRACAPA----------
% -----------------------------------------------
\imprimircapa % Está é Folha de Rosto
% -----------------------------------------------
\imprimirfolhaderosto % Esta é Contracapa
% -----------------------------------------------



% -----------------------------------------------
% -----------------ILUSTRAÇÕES-------------------
% -----------------------------------------------
\pdfbookmark[0]{\listfigurename}{lof}
\listoffigures*
\cleardoublepage
% -----------------------------------------------



% -----------------------------------------------
% ------------------TABELAS----------------------
% -----------------------------------------------
\pdfbookmark[0]{\listtablename}{lot}
\listoftables*
\cleardoublepage
% -----------------------------------------------



% -----------------------------------------------
% ------------ABREVIATURAS E SIGLAS--------------
% -----------------------------------------------
\begin{siglas}
  \item[ABNT] Associação Brasileira de Normas Técnicas
  \item[abnTeX] ABsurdas Normas para TeX
\end{siglas}
% -----------------------------------------------



% -----------------------------------------------
% ------------------SÍMBOLOS---------------------
% -----------------------------------------------
\begin{simbolos}
  \item[$ \Gamma $] Letra grega Gama
  \item[$ \Lambda $] Lambda
  \item[$ \zeta $] Letra grega minúscula zeta
  \item[$ \in $] Pertence
\end{simbolos}
% -----------------------------------------------



% -----------------------------------------------
% ------------------SUMÁRIO----------------------
% -----------------------------------------------
\pdfbookmark[0]{\contentsname}{toc}
\tableofcontents*
\cleardoublepage
% -----------------------------------------------



% -----------------------------------------------
% --------------ELEMENTOS TEXTUAIS---------------
% -----------------------------------------------
\textual




% ----------------------------------------------------------
% ----------------------Introdução--------------------------
% ----------------------------------------------------------
\chapter*[Introdução]{Introdução}
\addcontentsline{toc}{chapter}{Introdução}

\lipsum[1-2] % Gera texto de exemplo para preencher a página.
\lipsum[2] \footnote{\url{http://www.latex-project.org/lppl.txt}}.



% ----------------------------------------------------------
% -----------------Capitulo de textual----------------------  
% ----------------------------------------------------------
\chapter{Elementos textuais}

\section{Morbi orci nisl hendrerit}
\lipsum[1-2] % Gera texto de exemplo para preencher a página. 
\cite{abntex2classe}.

\section{Nam dui ligula fringilla euismod sodales, sollicitudin}
\lipsum[3-4] % Gera texto de exemplo para preencher a página. 

\section{Donec felis eratcongue non}
\lipsum[4-5] % Gera texto de exemplo para preencher a página. 



\chapter{Suspendisse vel felis}

\section{Praesent enim elit}
\lipsum[6-7] % Gera texto de exemplo para preencher a página. 
\cite{abntex2classe}.

\section{Curabitur et nunc}
\lipsum[8-9] % Gera texto de exemplo para preencher a página. 



\chapter{Cras non urna. Morbi eros pede}

\section{Nullam at lectus}
\lipsum[10-11] % Gera texto de exemplo para preencher a página. 
\cite{abntex2classe}.

\section{Praesent pretium}
\lipsum[12-13] % Gera texto de exemplo para preencher a página. 

\section{Quisque ornare tellus ullamcorper}
\lipsum[22-23] % Gera texto de exemplo para preencher a página. 



% ----------------------------------------------------------
% Finaliza a parte no bookmark do PDF
% para que se inicie o bookmark na raiz
% e adiciona espaço de parte no Sumário
% ----------------------------------------------------------
\phantompart




% ----------------------------------------------------------
% -----------------------Conclusão--------------------------
% ----------------------------------------------------------
\chapter*[Considerações Finais]{Considerações Finais}
\addcontentsline{toc}{chapter}{Considerações Finais}
\lipsum[31-33]



% ----------------------------------------------------------
% -----------------ELEMENTOS PÓS-TEXTUAIS-------------------
% ----------------------------------------------------------
\postextual

% ----------------------------------------------------------
% ----------------Referências bibliográficas----------------
% ----------------------------------------------------------
\chapter*{Referências}                                               % Insere o título
\markboth{Referências}{Referências}                                  % Título "Referências" no cabeçalho.
\addtocontents{toc}{\protect\vspace{\baselineskip}}                  % Adiciona uma linha.
\addcontentsline{toc}{chapter}{\protect\MakeUppercase{REFERÊNCIAS}}  % Nível 'chapter' para linha grossos.
\printbibliography[heading=none]                                     % Imprime a bibliografia.
% ----------------------------------------------------------



% ----------------------------------------------------------
% ---------------------APÊNDICE----------------------------- 
% ----------------------------------------------------------
\begin{apendicesenv}
\partapendices % Imprime uma página indicando o início dos apêndices

% ----------------------------------------------------------
\chapter{Quisque libero justo}

\section{Aliquam viverra arcu}
\lipsum[8-9] % Gera texto de exemplo para preencher a página. 

\section{Mauris consequat}
\lipsum[24-25] % Gera texto de exemplo para preencher a página. 


% ----------------------------------------------------------
\chapter{Nullam elementum}

\section{Quisque facilisis auctor sapien}
\lipsum[80-81] % Gera texto de exemplo para preencher a página. 

\section{Aliquam quis quam non metus}
\lipsum[34-35] % Gera texto de exemplo para preencher a página. 

\end{apendicesenv}
% ----------------------------------------------------------



% ----------------------------------------------------------
% -----------------------ANEXOS-----------------------------
% ----------------------------------------------------------
\begin{anexosenv}
\partanexos % Imprime uma página indicando o início dos anexos

% ----------------------------------------------------------
\chapter{Morbi ultrices rutrum lorem}

\section{Pellentesque sodales}
\lipsum[40-41] % Gera texto de exemplo para preencher a página. 

\section{Curabitur diam tortor}
\lipsum[54-55] % Gera texto de exemplo para preencher a página. 



% ----------------------------------------------------------
\chapter{Crasnatoque ridiculus}

\section{Mauris congue ligula}
\lipsum[70-71] % Gera texto de exemplo para preencher a página. 

\section{Aenean tincidunt laoreet dui}
\lipsum[14-15] % Gera texto de exemplo para preencher a página. 



% ----------------------------------------------------------
\chapter{Fusce facilisis lacinia duiod}
\lipsum[10-12]

\end{anexosenv}
% ----------------------------------------------------------




% ----------------------------------------------------------
% -------------------ÍNDICE REMISSIVO-----------------------
% ----------------------------------------------------------
\phantompart
\printindex

\end{document}
