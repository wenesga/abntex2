% -----------------------------------------------
% Modelo UFT para Trabalhos Acadêmicos
% Adaptado e mantido por Wenes (Libertário)
% Baseado no projeto abnTeX2, porém modificado para atender às diretrizes
% visuais e estruturais da Universidade Federal do Tocantins (UFT).
% Esta versão não é oficial e não substitui os modelos originais do abnTeX2.
% -----------------------------------------------
\documentclass[
	12pt,				% Tamanho da fonte
    oneside,            % Garante margens iguais (não alternadas)
	a4paper,			% tamanho do papel. 
	english,			% idioma adicional para hifenização
	french,				% idioma adicional para hifenização
	spanish,			% idioma adicional para hifenização
	brazilian,		    % o último idioma é o principal do documento
	]{abntex2}
% -----------------------------------------------



% -----------------------------------------------
% Pacotes básicos 
% -----------------------------------------------
\usepackage{lmodern}			% Usa a fonte Latin Modern			
\usepackage[T1]{fontenc}		% Selecao de codigos de fonte.
\usepackage[utf8]{inputenc}		% Codificacao do documento (conversão automática dos acentos)
\usepackage{indentfirst}		% Indenta o primeiro parágrafo de cada seção.
\usepackage{color}				% Controle das cores
\usepackage{graphicx}			% Inclusão de gráficos
\usepackage{microtype} 			% para melhorias de justificação
% -----------------------------------------------


		

% -----------------------------------------------
\usepackage{lipsum}				% para geração de dummy text
% -----------------------------------------------



% -----------------------------------------------
\usepackage[brazilian,hyperpageref]{backref}	 % Paginas com as citações na bibl
\usepackage[alf]{abntex2cite}	% Citações padrão ABNT
% -----------------------------------------------




% -----------------------------------------------
% Configurações do pacote backref
% Usado sem a opção hyperpageref de backref
\renewcommand{\backrefpagesname}{Citado na(s) página(s):~}
% Texto padrão antes do número das páginas
\renewcommand{\backref}{}
% Define os textos da citação
\renewcommand*{\backrefalt}[4]{
	\ifcase #1 %
		Nenhuma citação no texto.%
	\or
		Citado na página #2.%
	\else
		Citado #1 vezes nas páginas #2.%
	\fi}%
% -----------------------------------------------



% -----------------------------------------------
% Informações de dados para CAPA e FOLHA DE ROSTO
% -----------------------------------------------
\titulo{Modelo UFT para\\ Projeto de Pesquisa em \abnTeX}
\autor{Editado por Wenes Aquino}
\local{Palmas}
\data{2025, v-1.0}
\orientador{Editado por Wenes Aquino}
\coorientador{Editado por Wenes Aquino}

\instituicao{%
  Universidade Federal do Tocantins - UFT\\
  Campus Universitário de Palmas\\
  Curso de Licenciatura em Computação}
  
\tipotrabalho{Trabalho de Conclusão de Curso}

\preambulo{Trabalho de conclusão de curso apresentado como parte dos requisitos para avaliação na Universidade Federal do Tocantins (UFT). Este modelo foi adaptado e personalizado com ajustes visuais, estruturais e técnicos, servindo também para testar o espaço disponível e a formatação utilizada na contracapa deste documento.}
% -----------------------------------------------


% -----------------------------------------------
\makeatletter
\hypersetup{
    colorlinks=false,   % Desativa links coloridos
    pdfborder={0 0 0},  % Remove o retângulo ao redor dos links
}
\makeatother
% -----------------------------------------------





% -----------------------------------------------
% Posiciona figuras e tabelas no topo da página quando adicionadas sozinhas
% em um página em branco. Ver https://github.com/abntex/abntex2/issues/170
\makeatletter
\setlength{\@fptop}{5pt} % Set distance from top of page to first float
\makeatother
% -----------------------------------------------


% -----------------------------------------------
% Possibilita criação de Quadros e Lista de quadros.
% Ver https://github.com/abntex/abntex2/issues/176
%
\newcommand{\quadroname}{Quadro}
\newcommand{\listofquadrosname}{Lista de quadros}

\newfloat[chapter]{quadro}{loq}{\quadroname}
\newlistof{listofquadros}{loq}{\listofquadrosname}
\newlistentry{quadro}{loq}{0}

% configurações para atender às regras da ABNT
\setfloatadjustment{quadro}{\centering}
\counterwithout{quadro}{chapter}
\renewcommand{\cftquadroname}{\quadroname\space} 
\renewcommand*{\cftquadroaftersnum}{\hfill--\hfill}

\setfloatlocations{quadro}{hbtp} % Ver https://github.com/abntex/abntex2/issues/176
% -----------------------------------------------


% -----------------------------------------------
% Espaçamentos entre linhas e parágrafos 
% -----------------------------------------------
% O tamanho do parágrafo é dado por:
\setlength{\parindent}{1.3cm}
% Controle do espaçamento entre um parágrafo e outro:
\setlength{\parskip}{0.2cm}  % tente também \onelineskip
% -----------------------------------------------



% -----------------------------------------------
% compila o indice
% -----------------------------------------------
\makeindex
% -----------------------------------------------



% -----------------------------------------------
\begin{document}
% -----------------------------------------------
\selectlanguage{brazilian} % Idioma do documento 
\frenchspacing 
% -----------------------------------------------




% -----------------------------------------------
% ELEMENTOS PRÉ-TEXTUAIS
% -----------------------------------------------
% \pretextual



% -----------------------------------------------
\imprimircapa
% -----------------------------------------------
\imprimirfolhaderosto*
% -----------------------------------------------



% -----------------------------------------------
% Inserir a ficha bibliografica
% -----------------------------------------------
% Isto é um exemplo de Ficha Catalográfico, ou ``Dados internacionais de
% catalogação-na-publicação''. Você pode utilizar este modelo como referência. 
% Porém, provavelmente a biblioteca da sua universidade lhe fornecerá um PDF
% com a ficha catalográfico definitiva após a defesa do trabalho. Quando estiver
% com o documento, salve-o como PDF no diretório do seu projeto e substitua todo
% o conteúdo de implementação deste arquivo pelo comando abaixo:
% -----------------------------------------------
% \usepackage{pdfpages}		% necessário para comando \includepdf
% \begin{fichacatalografica}
%     \includepdf{fig_ficha_catalografica.pdf}
% \end{fichacatalografica}
% -----------------------------------------------
\begin{fichacatalografica}
	\sffamily
	\vspace*{\fill}					% Posição vertical
	\begin{center}					% Minipage Centralizado
	\fbox{\begin{minipage}[c][8cm]{13.5cm}		% Largura
	\small
	\imprimirautor
	%Sobrenome, Nome do autor
	
	\hspace{0.5cm} \imprimirtitulo  / \imprimirautor. --
	\imprimirlocal, \imprimirdata-
	
	\hspace{0.5cm} \thelastpage p. : il. (algumas color.) ; 30 cm.\\
	
	\hspace{0.5cm} \imprimirorientadorRotulo~\imprimirorientador\\
	
	\hspace{0.5cm}
    \parbox[t]{12.5cm}{\imprimirtipotrabalho~--~\imprimirinstituicao,
	\imprimirdata.}\\
	
	\hspace{0.5cm}
		1. Inteligência artificial.
		2. Pensamento computacional.
		3. Computação desplugado.
		I. Orientador.
		II. Universidade Federal do Tocantins.
		III. Campus Universitário de Palmas - Curso de Licenciatura em Computação.
		IV. Título. 			
	\end{minipage}}
	\end{center}
\end{fichacatalografica}
% -----------------------------------------------




% -----------------------------------------------
% --------------------ERRATA---------------------
% -----------------------------------------------
\begin{errata}
\textbf{Este é um exemplo de Errata (Elemento Opcional)}

\vspace{0.5cm}

A \textbf{Errata} é um elemento pré-textual opcional (em folha avulsa) utilizado para listar as correções necessárias no trabalho após a sua impressão. O seu objetivo é alertar o leitor sobre os erros que passaram pela revisão final.

\vspace{0.5cm}

Conforme a ABNT, a errata deve indicar de forma clara a folha (página), a linha, o texto incorreto e o texto corrigido.

\vspace{1cm}

\textbf{Exemplo de Estrutura para Correções}

\begin{center}
\begin{tabular}{|c|c|p{4cm}|p{4cm}|}
\hline
\textbf{Folha} & \textbf{Linha} & \textbf{Onde se lê (Texto Incorreto)} & \textbf{Leia-se (Texto Correto)} \\
\hline
15 & 8 & A tecnologia é Fotovoltáica. & A tecnologia é Fotovoltaica. \\
\hline
42 & 21 & Os algoritmos não corrigidos. & Os algoritmos corrigidos. \\
\hline
58 & 5 & A aplicação de computador. & A aplicação de computação. \\
\hline
\end{tabular}
\end{center}
\end{errata}





% ---
% Inserir folha de aprovação
% ---

% Isto é um exemplo de Folha de aprovação, elemento obrigatório da NBR
% 14724/2011 (seção 4.2.1.3). Você pode utilizar este modelo até a aprovação
% do trabalho. Após isso, substitua todo o conteúdo deste arquivo por uma
% imagem da página assinada pela banca com o comando abaixo:
%
% \begin{folhadeaprovacao}
% \includepdf{folhadeaprovacao_final.pdf}
% \end{folhadeaprovacao}
%
\begin{folhadeaprovacao}

  \begin{center}
    {\ABNTEXchapterfont\large\imprimirautor}

    \vspace*{\fill}\vspace*{\fill}
    \begin{center}
      \ABNTEXchapterfont\bfseries\Large\imprimirtitulo
    \end{center}
    \vspace*{\fill}
    
    \hspace{.45\textwidth}
    \begin{minipage}{.5\textwidth}
        \imprimirpreambulo
    \end{minipage}%
    \vspace*{\fill}
   \end{center}
        
   Trabalho aprovado. \imprimirlocal, 15 de novembro de 2025:

   \assinatura{\textbf{\imprimirorientador} \\ Orientador} 
   \assinatura{\textbf{Professor} \\ Convidado 1}
   \assinatura{\textbf{Professor} \\ Convidado 2}
   %\assinatura{\textbf{Professor} \\ Convidado 3}
   %\assinatura{\textbf{Professor} \\ Convidado 4}
      
   \begin{center}
    \vspace*{0.5cm}
    {\large\imprimirlocal}
    \par
    {\large\imprimirdata}
    \vspace*{1cm}
  \end{center}
  
\end{folhadeaprovacao}
% ---




% -----------------------------------------------
% Dedicatória
% -----------------------------------------------
\begin{dedicatoria}
   \vspace*{\fill}
   \centering
   \textit{Dedicado àqueles que compreendem o valor\\ 
        inestimável do conhecimento e da busca incessante\\ 
        pela verdade, alicerçando suas conquistas no mérito,\\ 
        na responsabilidade individual e na liberdade.\\
        Que esta jornada científica inspire a próxima\\ 
        geração de construtores, movidos pela lógica\\ 
        e pela convicção de que o progresso é fruto\\
        da iniciativa privada e do trabalho honesto.}
    \vspace*{\fill}
\end{dedicatoria}
% -----------------------------------------------





% -----------------------------------------------
% Agradecimentos
% -----------------------------------------------
\begin{agradecimentos}
À Universidade Federal do Tocantins (UFT) e à 
Universidade Aberta do Brasil (UAB), por fornecerem 
o ambiente acadêmico de excelência e o acesso ao 
conhecimento que tornaram esta jornada possível.
\par\vspace{0.5em}   

Ao Orientador \imprimirorientador, pela rigorosa 
e fundamental orientação técnica, que assegurou 
a precisão e a lógica deste trabalho.
\par\vspace{0.5em} 

Aos meus familiares e amigos mais próximos, 
pela compreensão e pelo suporte incondicional, 
que respeitaram a minha responsabilidade individual 
e o tempo dedicado à execução deste projeto.
\par\vspace{0.5em}  

A todos que, de forma direta ou indireta, 
contribuíram com o seu mérito e o seu 
trabalho para a conclusão desta pesquisa.
\end{agradecimentos}
% -----------------------------------------------



% -----------------------------------------------
% Epígrafe
% -----------------------------------------------
\begin{epigrafe}
    \vspace*{\fill}
    \begin{flushright}
        \textit{``A verdade é o fundamento que sustenta a mente livre.\\
        Onde há liberdade, nasce a responsabilidade;\\
        onde há responsabilidade, floresce o mérito;\\
        e onde o mérito prevalece, nenhum poder arbitrário\\
        consegue limitar o potencial do indivíduo.\\
        A busca pela liberdade não é apenas um ideal,\\
        mas um compromisso diário com a razão, a coragem\\
        e a defesa daquilo que torna cada ser humano único.''}\\[1em]
        $\forall$ Libertário \(\infty\) 
    \end{flushright}
\end{epigrafe}
% -----------------------------------------------

% ---
% RESUMOS
% ---

% resumo em português
\setlength{\absparsep}{18pt} % ajusta o espaçamento dos parágrafos do resumo
\begin{resumo}
 Segundo a \citeonline[3.1-3.2]{NBR6028:2003}, o resumo deve ressaltar o
 objetivo, o método, os resultados e as conclusões do documento. A ordem e a extensão
 destes itens dependem do tipo de resumo (informativo ou indicativo) e do
 tratamento que cada item recebe no documento original. O resumo deve ser
 precedido da referência do documento, com exceção do resumo inserido no
 próprio documento. (\ldots) As palavras-chave devem figurar logo abaixo do
 resumo, antecedidas da expressão Palavras-chave:, separadas entre si por
 ponto e finalizadas também por ponto.

 \textbf{Palavras-chave}: latex. abntex. editoração de texto.
\end{resumo}

% resumo em inglês
\begin{resumo}[Abstract]
 \begin{otherlanguage*}{english}
   This is the english abstract.

   \vspace{\onelineskip}
 
   \noindent 
   \textbf{Keywords}: latex. abntex. text editoration.
 \end{otherlanguage*}
\end{resumo}

% resumo em francês 
\begin{resumo}[Résumé]
 \begin{otherlanguage*}{french}
    Il s'agit d'un résumé en français.
 
   \textbf{Mots-clés}: latex. abntex. publication de textes.
 \end{otherlanguage*}
\end{resumo}

% resumo em espanhol
\begin{resumo}[Resumen]
 \begin{otherlanguage*}{spanish}
   Este es el resumen en español.
  
   \textbf{Palabras clave}: latex. abntex. publicación de textos.
 \end{otherlanguage*}
\end{resumo}
% ---

% ---
% inserir lista de ilustrações
% ---
\pdfbookmark[0]{\listfigurename}{lof}
\listoffigures*
\cleardoublepage
% ---

% ---
% inserir lista de quadros
% ---
\pdfbookmark[0]{\listofquadrosname}{loq}
\listofquadros*
\cleardoublepage
% ---

% ---
% inserir lista de tabelas
% ---
\pdfbookmark[0]{\listtablename}{lot}
\listoftables*
\cleardoublepage
% ---

% ---
% inserir lista de abreviaturas e siglas
% ---
\begin{siglas}
  \item[ABNT] Associação Brasileira de Normas Técnicas
  \item[abnTeX] ABsurdas Normas para TeX
\end{siglas}
% ---

% ---
% inserir lista de símbolos
% ---
\begin{simbolos}
  \item[$ \Gamma $] Letra grega Gama
  \item[$ \Lambda $] Lambda
  \item[$ \zeta $] Letra grega minúscula zeta
  \item[$ \in $] Pertence
\end{simbolos}
% ---

% ---
% inserir o sumario
% ---
\pdfbookmark[0]{\contentsname}{toc}
\tableofcontents*
\cleardoublepage
% ---



% ----------------------------------------------------------
% ELEMENTOS TEXTUAIS
% ----------------------------------------------------------
\textual

% ----------------------------------------------------------
% Introdução (exemplo de capítulo sem numeração, mas presente no Sumário)
% ----------------------------------------------------------
\chapter{Introdução}
% ----------------------------------------------------------

Este documento e seu código-fonte são exemplos de referência de uso da classe
\textsf{abntex2} e do pacote \textsf{abntex2cite}. O documento 
exemplifica a elaboração de trabalho acadêmico (tese, dissertação e outros do
gênero) produzido conforme a ABNT NBR 14724:2011 \emph{Informação e documentação
- Trabalhos acadêmicos - Apresentação}.

A expressão ``Modelo Canônico'' é utilizada para indicar que \abnTeX\ não é
modelo específico de nenhuma universidade ou instituição, mas que implementa tão
somente os requisitos das normas da ABNT. Uma lista completa das normas
observadas pelo \abnTeX\ é apresentada em \citeonline{abntex2classe}.

Sinta-se convidado a participar do projeto \abnTeX! Acesse o site do projeto em
\url{http://www.abntex.net.br/}. Também fique livre para conhecer,
estudar, alterar e redistribuir o trabalho do \abnTeX, desde que os arquivos
modificados tenham seus nomes alterados e que os créditos sejam dados aos
autores originais, nos termos da ``The \LaTeX\ Project Public
License''\footnote{\url{http://www.latex-project.org/lppl.txt}}.

Encorajamos que sejam realizadas customizações específicas deste exemplo para
universidades e outras instituições --- como capas, folha de aprovação, etc.
Porém, recomendamos que ao invés de se alterar diretamente os arquivos do
\abnTeX, distribua-se arquivos com as respectivas customizações.
Isso permite que futuras versões do \abnTeX~não se tornem automaticamente
incompatíveis com as customizações promovidas. Consulte
\citeonline{abntex2-wiki-como-customizar} para mais informações.

Este documento deve ser utilizado como complemento dos manuais do \abnTeX\ 
\cite{abntex2classe,abntex2cite,abntex2cite-alf} e da classe \textsf{memoir}
\cite{memoir}. 

Esperamos, sinceramente, que o \abnTeX\ aprimore a qualidade do trabalho que
você produzirá, de modo que o principal esforço seja concentrado no principal:
na contribuição científica.

Equipe \abnTeX 

Lauro César Araujo

% ----------------------------------------------------------
% PARTE
% ----------------------------------------------------------
\part{Preparação da pesquisa}
% ----------------------------------------------------------



\chapter{Conteúdos específicos do modelo de trabalho acadêmico}\label{cap_trabalho_academico}

\section{Quadros}

Este modelo vem com o ambiente \texttt{quadro} e impressão de Lista de quadros 
configurados por padrão. Verifique um exemplo de utilização:

\begin{quadro}[htb]
\caption{\label{quadro_exemplo}Exemplo de quadro}
\begin{tabular}{|c|c|c|c|}
	\hline
	\textbf{Pessoa} & \textbf{Idade} & \textbf{Peso} & \textbf{Altura} \\ \hline
	Marcos & 26    & 68   & 178    \\ \hline
	Ivone  & 22    & 57   & 162    \\ \hline
	...    & ...   & ...  & ...    \\ \hline
	Sueli  & 40    & 65   & 153    \\ \hline
\end{tabular}
\fonte{Autor.}
\end{quadro}

Este parágrafo apresenta como referenciar o quadro no texto, requisito
obrigatório da ABNT. 
Primeira opção, utilizando \texttt{autoref}: Ver o \autoref{quadro_exemplo}. 
Segunda opção, utilizando  \texttt{ref}: Ver o Quadro \ref{quadro_exemplo}.

% ----------------------------------------------------------
% PARTE
% ----------------------------------------------------------
\part{Referenciais teóricos}
% ----------------------------------------------------------

% ---
% Capitulo de revisão de literatura
% ---
\chapter{Lorem ipsum dolor sit amet}
% ---

% ---
\section{Aliquam vestibulum fringilla lorem}
% ---

\lipsum[1]

\lipsum[2-3]

% ----------------------------------------------------------
% PARTE
% ----------------------------------------------------------
\part{Resultados}
% ----------------------------------------------------------

% ---
% primeiro capitulo de Resultados
% ---
\chapter{Lectus lobortis condimentum}
% ---

% ---
\section{Vestibulum ante ipsum primis in faucibus orci luctus et ultrices
posuere cubilia Curae}
% ---

\lipsum[21-22]

% ---
% segundo capitulo de Resultados
% ---
\chapter{Nam sed tellus sit amet lectus urna ullamcorper tristique interdum
elementum}
% ---

% ---
\section{Pellentesque sit amet pede ac sem eleifend consectetuer}
% ---

\lipsum[24]

% ----------------------------------------------------------
% Finaliza a parte no bookmark do PDF
% para que se inicie o bookmark na raiz
% e adiciona espaço de parte no Sumário
% ----------------------------------------------------------
\phantompart

% ---
% Conclusão
% ---
\chapter{Conclusão}
% ---

\lipsum[31-33]

% ----------------------------------------------------------
% ELEMENTOS PÓS-TEXTUAIS
% ----------------------------------------------------------
\postextual
% ----------------------------------------------------------

% ----------------------------------------------------------
% Referências bibliográficas
% ----------------------------------------------------------
\bibliography{Referencias}

% ----------------------------------------------------------
% Glossário
% ----------------------------------------------------------
%
% Consulte o manual da classe abntex2 para orientações sobre o glossário.
%
%\glossary

% ----------------------------------------------------------
% Apêndices
% ----------------------------------------------------------

% ---
% Inicia os apêndices
% ---
\begin{apendicesenv}

% Imprime uma página indicando o início dos apêndices
\partapendices

% ----------------------------------------------------------
\chapter{Quisque libero justo}
% ----------------------------------------------------------

\lipsum[50]

% ----------------------------------------------------------
\chapter{Nullam elementum urna vel imperdiet sodales elit ipsum pharetra ligula
ac pretium ante justo a nulla curabitur tristique arcu eu metus}
% ----------------------------------------------------------
\lipsum[55-57]

\end{apendicesenv}
% ---


% ----------------------------------------------------------
% Anexos
% ----------------------------------------------------------

% ---
% Inicia os anexos
% ---
\begin{anexosenv}

% Imprime uma página indicando o início dos anexos
\partanexos

% ---
\chapter{Morbi ultrices rutrum lorem.}
% ---
\lipsum[30]

% ---
\chapter{Cras non urna sed feugiat cum sociis natoque penatibus et magnis dis
parturient montes nascetur ridiculus mus}
% ---

\lipsum[31]

% ---
\chapter{Fusce facilisis lacinia dui}
% ---

\lipsum[32]

\end{anexosenv}

%---------------------------------------------------------------------
% INDICE REMISSIVO
%---------------------------------------------------------------------
\phantompart
\printindex
%---------------------------------------------------------------------

\end{document} 