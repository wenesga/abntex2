\documentclass[aspectratio=169]{beamer}
\usepackage{beamerthemeWenesMinimal}
\usepackage[brazil]{babel}
\usepackage{graphicx}





\title{\textsc{Demonstração do Tema WenesMinimal}}
\author{\textsc{Wenes Gomes Aquino}}
\date{2025}










\begin{document}

% --------------------------------------------------
% TÍTULO
% --------------------------------------------------

\begin{frame}
  \titlepage
\end{frame}

\WenesSectionPage{Exemplos de Layout}

% --------------------------------------------------
% 1) DUAS COLUNAS LADO A LADO
% --------------------------------------------------

\begin{frame}{Duas Colunas}
\begin{columns}[T,onlytextwidth]

  \column{0.48\textwidth}
  \begin{block}{\textsc{Coluna Esquerda}}
    Conteúdo organizado, listas, parágrafos, etc.
    \begin{itemize}
      \item Item A
      \item Item B
      \item Item C
    \end{itemize}
  \end{block}

  \column{0.48\textwidth}
  \begin{block}{\textsc{Coluna Direita}}
    Texto complementando a explicação:
    \pause % animação opcional
    \begin{itemize}
      \item Informação 1
      \item Informação 2
      \item Informação 3
    \end{itemize}
  \end{block}

\end{columns}
\end{frame}

% --------------------------------------------------
% 2) CAIXAS PARALELAS (CARDS)
% --------------------------------------------------

\begin{frame}{Cards Lado a Lado}
\begin{columns}[T,onlytextwidth]

  \column{0.48\textwidth}
    \begin{block}{\textsc{Card A}}
      Ideal para pontos fortes ou conceitos.
    \end{block}

  \column{0.48\textwidth}
    \begin{block}{\textsc{Card B}}
      Excelente para contraste visual.
    \end{block}

\end{columns}
\end{frame}

% --------------------------------------------------
% 3) QUADRO COM BORDA (ESTILO "BOX")
% --------------------------------------------------

\begin{frame}{Box com Borda}
\setbeamercolor{mybox}{fg=wm-dark,bg=wm-white}
\begin{beamercolorbox}[wd=\textwidth,sep=1ex,rounded=true,shadow=false]{mybox}
  \Large \textbf{Este é um box com borda e fundo branco.}
  \medskip

  Útil para destacar avisos, notas e definições.
\end{beamercolorbox}
\end{frame}

% --------------------------------------------------
% 4) BOX DE FUNDO ESCURO
% --------------------------------------------------

\begin{frame}{Box Forte}
\setbeamercolor{mydarkbox}{fg=wm-white,bg=wm-blue}
\begin{beamercolorbox}[wd=.9\textwidth,sep=1.5ex,rounded=true]{mydarkbox}
  \Large \textbf{Box de Destaque}
\end{beamercolorbox}

\bigskip
Use para chamar a atenção em conceitos importantes.
\end{frame}

% --------------------------------------------------
% 5) SLIDE DE IMAGEM E TEXTO
% --------------------------------------------------

\begin{frame}{Imagem + Texto}
\begin{columns}[T,onlytextwidth]

  \column{0.48\textwidth}
    \includegraphics[width=\textwidth]{example-image}

  \column{0.48\textwidth}
    \Large
    Visual limpo com espaço para descrição.
    \bigskip

    \begin{itemize}
      \item Design suave
      \item Boa hierarquia visual
      \item Ideal para apresentações modernas
    \end{itemize}

\end{columns}
\end{frame}

% --------------------------------------------------
% 6) QUOTE CENTRAL (FULL)
% --------------------------------------------------

\begin{frame}[plain]
  \WenesBigQuote[Paul Rand]{O design é o embaixador silencioso da sua marca.}
\end{frame}

% --------------------------------------------------
% 7) GRID 2x2
% --------------------------------------------------

\begin{frame}{Grade 2×2}
\begin{columns}[T,onlytextwidth]

  \column{0.48\textwidth}
    \begin{block}{\textsc{A}}
      Parte A da grade.
    \end{block}
    \begin{block}{\textsc{B}}
      Parte B da grade.
    \end{block}

  \column{0.48\textwidth}
    \begin{block}{\textsc{C}}
      Parte C da grade.
    \end{block}
    \begin{block}{\textsc{D}}
      Parte D da grade.
    \end{block}

\end{columns}
\end{frame}

% --------------------------------------------------
% 8) COMPARAÇÃO VISUAL (VS)
% --------------------------------------------------

\begin{frame}{Comparação: A vs B}
\begin{columns}[T,onlytextwidth]

  \column{0.48\textwidth}
    \begin{block}{\textsc{Opção A}}
      \begin{itemize}
        \item Simples
        \item Direta
        \item Leve
      \end{itemize}
    \end{block}

  \column{0.1\textwidth}
    \centering
    {\Huge \textbf{VS}}

  \column{0.48\textwidth}
    \begin{block}{\textsc{Opção B}}
      \begin{itemize}
        \item Completa
        \item Expansiva
        \item Detalhada
      \end{itemize}
    \end{block}

\end{columns}
\end{frame}

% --------------------------------------------------
% 9) LISTA NUMERADA ESTILIZADA
% --------------------------------------------------

\begin{frame}{Lista Numerada}
\Large
\begin{enumerate}
  \item Primeira etapa
  \item Segunda fase
  \item Terceiro ponto
\end{enumerate}
\end{frame}

% --------------------------------------------------
% 10) SLIDE DE DESTAQUE TOTAL
% --------------------------------------------------

\begin{frame}[plain]
  \centering
  {\Huge\bfseries\scshape
  Destaque Supremo}
  \bigskip

  {\large Clareza + Simplicidade + Força Visual}
\end{frame}

% --------------------------------------------------
% 11) SEÇÃO DELUXE
% --------------------------------------------------

\WenesSectionPage{Encerramento}

\begin{frame}{Obrigado!}
  \centering
  {\Large Perguntas?}
\end{frame}




\begin{frame}{Linha do Tempo}
\Large

\begin{itemize}
  \item[2019] Início do projeto.
  \item[2020] Expansão e testes.
  \item[2022] Implementação completa.
  \item[2025] Nova fase estratégica.
\end{itemize}

\end{frame}



\begin{frame}{Processo em 3 Etapas}
\Large

\begin{enumerate}
  \item \textbf{Planejamento} — definição dos objetivos.
  \item \textbf{Execução} — implementação das ações.
  \item \textbf{Avaliação} — análise dos resultados.
\end{enumerate}

\end{frame}



\begin{frame}{Impacto}
\centering
{\Huge\bfseries 98\%}
\bigskip

{\Large Taxa de satisfação dos usuários.}
\end{frame}


\begin{frame}{Checklist}
\Large

\begin{itemize}
  \item[\checkmark] Item verificado
  \item[\checkmark] Requisito cumprido
  \item[\checkmark] Documentação validada
\end{itemize}

\end{frame}

\begin{frame}{Tabela Comparativa}

\begin{table}[h]
\centering
\begin{tabular}{lccc}
\textbf{Recurso} & \textbf{Plano A} & \textbf{Plano B} & \textbf{Plano C} \\ \hline
Armazenamento     & 10 GB & 50 GB & 1 TB \\
Suporte           & Básico & Completo & Premium \\
Relatórios        & — & Parcial & Total \\
\end{tabular}
\end{table}

\end{frame}

\begin{frame}{Indicadores}
\begin{columns}[T,onlytextwidth]

\column{0.32\textwidth}
\centering {\Huge 42\%} \\ Crescimento

\column{0.32\textwidth}
\centering {\Huge 120k} \\ Usuários

\column{0.32\textwidth}
\centering {\Huge 9.8} \\ Avaliações

\end{columns}
\end{frame}
\begin{frame}{Depoimento}

\begin{columns}[T,onlytextwidth]

\column{0.25\textwidth}
\includegraphics[width=\textwidth]{example-image-a}

\column{0.72\textwidth}
\Large \textit{“A melhor solução que usamos nos últimos 10 anos.”}
\bigskip

\hfill — \textsc{Especialista do Setor}

\end{columns}

\end{frame}
\begin{frame}
\centering
{\Huge\bfseries\scshape Título Forte}
\bigskip
{\Large Subtítulo Explicativo e Elegante}
\end{frame}

\begin{frame}[plain]
\begin{picture}(0,0)
  \put(-20,-140){\includegraphics[width=\paperwidth]{example-image}}
\end{picture}

\vspace*{2cm}
\color{white}
\Huge\bfseries Texto Sobreposto

\end{frame}
\begin{frame}{Três Cards}

\begin{columns}[T,onlytextwidth]

  \column{0.32\textwidth}
    \begin{block}{\textsc{Card 1}} Conteúdo leve. \end{block}

  \column{0.32\textwidth}
    \begin{block}{\textsc{Card 2}} Explicativo. \end{block}

  \column{0.32\textwidth}
    \begin{block}{\textsc{Card 3}} Informativo. \end{block}

\end{columns}
\end{frame}









\begin{frame}[plain]
\centering
{\Huge\bfseries Obrigado!}
\bigskip
{\Large \textsc{WenesMinimal Theme}}
\bigskip

\textcolor{wm-muted}{2025}
\end{frame}








\end{document}
